\documentclass[a4paper,11pt]{article}
\usepackage[utf8]{inputenc}
\usepackage[ngerman]{babel}
\usepackage{microtype}
\usepackage{cmbright}
\usepackage[T1]{fontenc}

%opening
\title{WSUS - Basis}
\author{Markus Reichl}

\begin{document}

\maketitle

\section{Vorbereitung}
\subsection{Szenario}
\subsubsection{Einfache WSUS-Bereitstellung}
Die einfachste WSUS-Bereitstellung besteht aus einem Server innerhalb der Unternehmensfirewall, der Updates für Clientcomputer in einem privaten Intranet verarbeitet. 
Der WSUS-Server stellt eine Verbindung mit Microsoft Update her, um Updates herunterzuladen.
\subsubsection{Mehrere WSUS Server}
Administratoren können mehrere Server mit WSUS bereitstellen, die alle Inhalte innerhalb des Intranets der Organisation synchronisieren.
\subsubsection{Nicht verbundener WSUS Server}
Wenn der Zugriff auf das Internet aufgrund von Unternehmensrichtlinien oder aus anderen Gründen eingeschränkt ist, können Administratoren einen internen Server für WSUS einrichten.
\subsubsection{Serverhierarchien}
Da es möglich ist, einen WSUS-Server nicht mit Microsoft Update, sondern mit einem anderen WSUS-Server zu synchronisieren, benötigt man nur einen mit Microsoft Update verbundenen WSUS-Server.

\subsection{Datenspeicherung}
Windows Server Update Services (WSUS) verwendet zwei Arten von Speichersystemen: eine Datenbank zum Speichern der WSUS-Konfiguration und Updatemetadaten und ein optionales lokales Dateisystem zum Speichern von Updatedateien.
\subsubsection{WSUS Datenbank}
WSUS erfordert eine Datenbank für jeden WSUS-Server. WSUS unterstützt – mit einigen Einschränkungen – die Verwendung einer Datenbank, die sich auf einem anderen Computer als dem WSUS-Server befindet.
Diese Datenbank kann entweder eine interne Windows Datenbank, oder ein SQL Server sein.
\subsubsection{WSUS Updatespeicher}
Beim Synchronisieren von Updates auf den WSUS-Server werden die Metadaten- und Updatedateien an zwei separaten Speicherorten gespeichert. Metadaten werden in der WSUS-Datenbank gespeichert. 
Updatedateien können auf dem WSUS-Server oder auf Microsoft Update-Servern gespeichert werden, je nachdem, wie die Optionen für die Synchronisierung konfiguriert wurde.

\section{Installation}
\begin{itemize}
 \item Navigation auf ``Verwalten'' und dann auf ``Rollen und Features hinzufügen''.
 \item Installationstyp ``Rollenbasierte oder featurebasierte Installation'' ausw\"ahlen.
 \item Auswahl des Zielservers, sowie des Ortes auf dem die WSUS-Serverrolle installiert werden soll.
 \item Auswahl der Rolle Windows Server Update Services (WSUS).
 \item Für Windows Server Update Services erforderliche Features hinzugefügen.
 \item Standardeinstellungen beibehalten und durchklicken.
  \subitem Standardkonfiguration der Webserverrolle erforderlich.
  \subitem Datenbanktyp auswählen und eventuell Optionen aktivieren.
 \item Gültigen Speicherort zum Speichern der Updates angeben. Empfohlen ist daf\"ur ein eigens angelegter Ordner.
 \item Informationen zur Webserverrolle (IIS) \"uberpr\"ufun und mit Standardeinstellungen durchklicken.
 \item Überprüfen der Installationsauswahl und Bestätigung der ausgewählten Optionen.
 \item Wenn erforderlich das System neustarten.
\end{itemize}
\end{document}