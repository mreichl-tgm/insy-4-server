\documentclass{preset/school}

\subject{INSY}
\title{Windows Server}
\subtitle{WSUS Konfiguration}

\author{Markus Reichl}

\begin{document}

\maketitle
\tableofcontents
% \begin{thebibliography}{9}
% \end{thebibliography}

\newpage
\section{Vorbereitung}
In der Standardkonfiguration ruft WSUS Updates von Microsoft Update ab. Wenn sich ein Proxyserver im Netzwerk befindet, kann WSUS zur Verwendung des Proxyservers konfiguriert werden.
In diesem Fall sollte soller Zugriff auf diesen Server, sowie dessen Firewall gegeben sein.\\
Updates können auch in Netzwerke importiert werden, die nicht mit dem Internet verbunden sind.

\section{Konfigurieren von Netzwerkverbindungen}
\paragraph{Updates}
Zu Beginn muss angegeben werden wie der Server Updates abruft (von Microsoft Update oder von einem anderen WSUS-Server).
\paragraph{Proxy}
Wenn WSUS einen Proxyserver für den Internetzugriff verwenden muss, müssen die Proxyeinstellungen im WSUS-Server konfiguriert werden.
\paragraph{Firewall}
Wenn sich WSUS hinter einer Firewall befindet, müssen einige zusätzliche Schritte auf dem Edgegerät ausgeführt werden, um den WSUS-Datenverkehr korrekt zuzulassen.

\subsection{Verbindung zum Internet}
Der WSUS-Server verwendet standardmäßig den Port 443 für das HTTPS-Protokoll, um Updates von Microsoft Update herunterzuladen. Die meisten Firewalls lassen diese Art von Datenverkehr zu.

\subsubsection{Mit Firewall}
Da WSUS den gesamten Netzwerkdatenverkehr initiiert, muss die Windows-Firewall auf dem WSUS-Server nicht konfiguriert werden.
Obwohl die Verbindung zwischen Microsoft Update und WSUS erfordert, dass die Ports 80 und 443 geöffnet sind, können mehrere WSUS-Server zur Synchronisierung mit einem benutzerdefinierten Port eingestellt werden.

\subsubsection{Verbindung zwischen Clients und WSUS-Servern}
Die Überwachungsschnittstellen und Ports werden auf den IIS-Websites für WSUS und in allen Gruppenrichtlinieneinstellungen zur Konfiguration von Clientcomputern konfiguriert.

\paragraph{Konfigurieren der Proxyserver}
Werden Proxyserver verwendet, müssen die Proxyserver die Protokolle HTTP und SSL unterstützen und Standardauthentifizierung oder Windows-Authentifizierung verwenden.
Es ist auch möglich, dass mehrere Proxyserver nur einen Teilaspekt erfüllen.

\subparagraph{Einrichten von WSUS zur Verwendung von Proxyservern}
\begin{enumerate}
    \item Anmeldung am Server als Administrator.
    \item Installation der WSUS-Serverrolle. Hier wird noch kein Proxyserver angegeben.
    \item Eingabeaufforderung (cmd.exe) als Administrator öffnen.
    \item Wechseln im Eingabeaufforderungsfenster zum Ordner\\
    \texttt{C:\textbackslash{}Programme\textbackslash{}Update Services\textbackslash{}Tools}.
    \item \texttt{wsusutil ConfigureSSLProxy [<Proxyserver Proxyport>] –enable} eingeben
    \item Öffnen der WSUS-Verwaltungskonsole.
    \item Erweitern des Servernamen, und klicken auf Optionen.
    \item Klicken auf Updatequelle und Proxyserver und anschließend auf die Registerkarte Proxyserver.
    \item Ändern oder Hinzufügen eines Proxyservers zur WSUS-Konfiguration.
    \item Aktivieren des Kontrollkästchens für ``Proxyserver für die Synchronisierung verwenden''.
\end{enumerate}

\section{Konfigurieren von WSUS}
\begin{enumerate}
    \item Klicken im Navigationsbereich des Server-Managers auf Dashboard, auf Tools und dann auf Windows Server Update Services.
    \item Der Anleitung des Assistenten folgen und wenn nötig Einstellungen vornehmen.
    \item Auf der Seite Upstreamserver auswählen stehen zwei Optionen zur Verfügung: ``Updates mit Microsoft Update synchronisieren'' und ``Von einem anderen Windows Server Update Services-Server synchronisieren''.\\
    Bei letzterem müssen Servername und Port angegeben werden, über welche der Server mit dem Upstreamserver kommuniziert und nötige Änderungen an den Kontrollkästchen vorgenommen werden.
    \item Nachdem die entsprechenden Optionen ausgewählt wurden mit Weiter fortfahren.
    \item Mit Upstreamserver verbinden klicken und auf Verbindung starten.
    \item Auf der Seite Produkte auswählen können Produkte gewählt werden, für welche Updates heruntergeladen werden sollen.
    \item Auswahl von Produktkategorien oder bestimmten Produkten.
    \item Auswahl von Produktoptionen für die Bereitstellung.
    \item Auswahl von Update-Klassifizierungen.
    \item Auf der Seite ``Synchronisierungszeitplan festlegen'' auswählen, ob die Synchronisierung manuell oder automatisch ausgeführt werden soll.
    \item Bei Auswahl von Manuell synchronisieren muss die Synchronisierung über die WSUS-Verwaltungskonsole gestartet werden.
    \item Bei Auswahl von Automatisch synchronisieren führt der WSUS-Server die Synchronisierung in festgelegten Intervallen aus.
    \item Zeit für Erste Synchronisierung festlegen und Synchronisierungen pro Tag angeben, die dieser Server ausführen soll.
\end{enumerate}

Die Synchronisation kann nun direkt gestartet werden, indem Sie das Kontrollkästchen Erstsynchronisierung starten aktiviert wird.
Ansonsten muss die Erstsynchronisierung über die WSUS-Verwaltungskonsole ausgeführt werden.
\\\vspace{0.5 em}
Damit ist die grundlegende WSUS-Konfiguration abgeschlossen.

\section{Konfigurieren von Computergruppen}
Mithilfe von Computergruppen können Updates getestet und gezielt für bestimmte Computer angewendet werden.
\\\vspace{0.5 em}
Standardmäßig fügt der WSUS-Server jeden Client bei der ersten Verbindungsherstellung einer der zwei Standardcomputergruppen, ``Alle Computer'' oder ``Nicht zugewiesene Computer'' hinzu.
Es wird empfohlen, mindestens eine weitere Computergruppe zu erstellen, mit der Updates getestet werden können, bevor sie auf anderen Computern bereitgestellt werden.

\subsection{Erstellen einer Computergruppe}
\begin{enumerate}
    \item Erweitern des Eintrags Computer unter Update Services in der Konsole und klick auf Alle Computer, sowie auf Computergruppe hinzufügen.
    \item Im Dialogfeld Computergruppe hinzufügen den Namen der neuen Gruppe in das Feld Name eintragen, und Hinzufügen klicken.
    \item Klicken auf Computer, und anschließende Auswahl der Computer.
    \item Klicken auf die im vorhergehenden Schritt ausgewählten Computernamen, und auf Mitgliedschaft ändern.
    \item Im Dialogfeld Gruppenmitgliedschaft für Computer festlegen die erstellte Testgruppe auswählen und bestätigen.
\end{enumerate}

\newpage
\section{Konfigurieren von Clientupdates}
Beim WSUS-Setup wird IIS automatisch so konfiguriert, dass die aktuelle Version von ``Automatische Updates'' an jeden Clientcomputer verteilt wird, der eine Verbindung mit dem WSUS-Server herstellt.
Die für ``Automatische Updates'' am besten geeignete Konfiguration hängt von der Netzwerkumgebung ab.
\\\vspace{0.5 em}
Wenn ein Active Directory-Verzeichnisdienst verwendet wird, kann ein vorhandenes Gruppenrichtlinienobjekt (GPO) verwendet werden oder ein neues erstellt werden.
\\\vspace{0.5 em}
In einer Umgebung ohne Active Directory sollte die Konfiguration über den Editor für lokale Gruppenrichtlinien vorgenommen werden und anschließend auf den Client und den WSUS-Server verweisen.
Über ``Konfigurieren von Automatische Updates'' in ``Gruppenrichtlinie'' können die Updates dann eingestellt werden.

\section{Konfigurieren von Computergruppen}
Nach der Konfiguration der Clientupdates können auf die Computergruppen konfiguriert werden.
\\\vspace{0.5 em}
Mit Active Directory können mehrere Computer gleichzeitig konfiguriert werden, indem einfach die GPO mit WSUS-Einstellungen versehen wird.
Es wird empfohlen, ein neues GPO zu erstellen, das nur WSUS-Einstellungen enthält und dieses mit einem geeigneten Active Directory-Container zu verbinden.

\paragraph{Aktivieren von WSUS über ein Domänen-GPO}
\begin{enumerate}
    \item Navigieren in der Gruppenrichtlinien-Verwaltungskonsole zum GPO, in dem WSUS konfiguriert werden soll und Bearbeiten klicken.
    \item Erweitern von Computerkonfiguration, Richtlinien, Administrative Vorlagen und Windows-Komponenten, und klicken auf Windows Update.
    \item Automatische Updates konfigurieren wählen und die Richtlinie Automatische Updates konfigurieren wird geöffnet.
    \item Klicken auf Aktiviert und wählen einer der folgenden Optionen:\\
    ``Vor Herunterladen und Installation benachrichtigen'', ``Autom. Herunterladen, aber vor Installation benachrichtigen'', ``Autom. Herunterladen und laut Zeitplan installieren'', ``Lokalen Administrator ermöglichen, Einstellung auszuwählen''.
    \item Clientseitige Zielzuordnung aktivieren und dann Aktiviert auf aus setzten.
    \item Namen der WSUS-Computergruppe angeben, welcher der Computer hinzufügt werden soll.
\end{enumerate}
Nach dem Einrichten eines Clients dauert es einige Minuten, bis der Computer auf der Seite Computer in der WSUS-Verwaltungskonsole angezeigt wird.

\section{Abschluss der IIS-Konfiguration}
Standardmäßig ist der anonyme Lesezugriff für die Standardwebsite und alle neuen IIS-Websites aktiviert.
Andernfalls müssen die Schritte für die entsprechende Version von IIS ausgeführt werden, um anonymen Lesezugriff zu aktivieren.

\section{Konfigurieren eines Signaturzertifikats}
WSUS bietet die Möglichkeit, benutzerdefinierte Updatepakete für Produkte von Microsoft und andere Produkte zu veröffentlichen.
WSUS kann diese benutzerdefinierten Updatepakete automatisch mit einem Zertifikat signieren. Dazu muss ein Paketsignaturzertifikat auf dem WSUS-Server installiert werden.
\\\vspace{0.5 em}
Der private Schlüssel muss auf dem WSUS-Server installiert sein, und der öffentliche Schlüssel muss explizit im Speicher für vertrauenswürdige Zertifikate auf allen Client-PCs und Servern installiert werden, die benutzerdefinierte signierte Updates erhalten sollen.
\\\vspace{0.5 em}
Wenn das selbstsignierte Zertifikat abläuft oder sich das Ablaufdatum nähert, protokolliert WSUS entsprechende Ereignisse im Ereignisprotokoll.

\end{document}
