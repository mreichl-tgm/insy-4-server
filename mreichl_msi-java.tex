\documentclass{preset/school}

\subject{INSY}
\title{Windows Server}
\subtitle{MSI Verteilung am Beispiel Java}

\author{Markus Reichl}

\begin{document}

\maketitle

\section{Einführung}
Die Windows-Pakete für die Java Runtime werden als .exe und nicht als .msi geliefert, wodurch sie prinzipiell nicht für die automatische Verteilung per Gruppenrichtlinie geeignet sind.
Im Fall von Java gibt es jedoch eine Option namens \texttt{msiexec.exe} an.

\section{Wahl der Downloadquelle}
Die Standard Installation von Java enthält zusätzliche Software-Komponenten, welche aber nur zu Werbezwecken dienen.
Der Benutzer kann diese zwar interaktiv abwählen, aber bei der Bereitstellung des Paketes würden diese mit installiert werden.
Auf der Übersichtsseite für die Java-Downloads ist daher der Link \texttt{Java SE for Business} zu wählen.
Auch wenn 32-Bit-System mittlerweile selten sind sollten diese nicht benachteiligt werden, also wird der Link zu diesem auch angegeben.

\section{Setup}
Um das verpackte MSI-Paket der Installation zu erhalten, genügt es, die Software-Installation einmal interaktiv zu starten.
Die MSI-Datei wird dann unter\\ \texttt{\%UserProfile\%\textbackslash{}AppData\textbackslash{}LocalLow\textbackslash{}Sun\textbackslash{}Java} in einem Unterverzeichnis mit der jeweiligen Versionsbezeichnung gespeichert.
Diese kann dann zusammen mit der Datendatei \texttt{Data1.cab} in die Bereitstellungsfreigabe kopiert werden. Danach erstellt man einfach eine Gruppenrichtlinie (siehe WSUS Dokumentation) zur Verteilung der Software.
\\
Die alte Java Runtime wird bei der Installation entfernt, dieser Schritt muss also nicht manuell vorgenommen werden.

\end{document}
