\documentclass[a4paper,11pt]{article}
\usepackage[utf8]{inputenc}
\usepackage[ngerman]{babel}
\usepackage{microtype}
\usepackage{cmbright}
\usepackage[T1]{fontenc}

%opening
\title{Windows Bereitstellungsdienste II}
\author{Markus Reichl}

\begin{document}

\maketitle

\section{Hinzuf\"ugen von Start- und Installationsabbildern}
\subsection{Startabbilder}
Bei einem Startabbild handelt es sich um Windows PE-Abbilder, mit dem einen Clientcomputer gestartet, eine Installation des Betriebssystems ausführen. 
In den meisten Szenarien sollte die Datei "Boot.wim" von den Installationsmedien (im Ordner "Sources") verwendet werden. 
Die Datei "Boot.wim" enthält Windows PE und den Windows-Bereitstellungsdiensteclient.

\subsection{Installationsabbilder}
Bei einem Installationsabbild handelt es sich um das Betriebssystem-Images, die auf dem Clientcomputer bereitgestellt. 
Diese k\"onnen auch durch die Install.wim-Datei von den Installationsmedien (im Ordner "Sources"), oder durch ein eigenes Installationsabbild erstellt werden (anhand der Schritte im ``Erstellen von benutzerdefinierten Installationsabbildern'').

\subsection{Vorraussetzungen zur Installation}
\begin{itemize}
 \item Der Clientcomputer muss einen PXE-Start ausführen.
 \item Der Clientcomputer muss mindestens 512 MB RAM verfügen, die Mindestmenge an Arbeitsspeicher für die Verwendung von Windows PE.
 \item Der Client muss das System für das Betriebssystem des Installationsabbilds erfüllen.
 \item Ein lokales Benutzerkonto muss ein, die auf der Windows-Bereitstellungsdienste-Server erstellt werden.
\end{itemize}

\subsection{Installation von Installationsabbildern (Am Empf\"anger)}
\begin{itemize}
 \item Konfigurieren des PXE-Starts und Startreihenfolge festlegen, sodass der Computer zuerst vom Netzwerk aus gestartet wird.
 \item Neustarten und bei Aufforderung F12 dr\"ucken um einen Netzwerkstart auszulösen.
 \item Auswahl des gewünschten Startabbilds aus dem Startmenü.
 \item Beim Verbinden des WDS-ServersAuthentifizierungsdialogfelds, das lokale Benutzerkonto und ein Kennwort eingeben.
 \item \"Ubliches Vorgehen bei der Installation des Betriebssystems.
\end{itemize}

\subsection{Erstellen von benutzerdefinierten Installationsabbildern}
\subsubsection{Aufzeichnungsabbild}
Zum Erstellen eines Installationsabbilds muss zunächst ein Aufzeichnungsabbild erstellt werden. 
Bei einem Aufzeichnungsabbild handelt es sich um Startabbilder, mit dem einen Clientcomputer gestartet, um das Betriebssystem in einer WIM-Datei aufzuzeichnen.

\begin{itemize}
 \item Erweitern der Windows-Bereitstellungsdienste den Knoten Startabbilder.
 \item Klicken auf das Bild, um es als Aufzeichnungsabbild zu verwenden.
 \item Klicken auf Aufzeichnungsabbild erstellen.
 \item Angabe von Image-Beschreibung, Speicherort und Dateinamen ein, die eine lokale Kopie der Datei gespeichert werden soll.
 \item Ein Bild hinzufügen.
 \item Speicherort der Windows-Abbilddatei, die die Bilder enthält angeben.
 \item Abbildname und Abbildbeschreibung angeben.
 \item Bis zum Ende durchklicken.
\end{itemize}

\subsubsection{Installationsabbild}
\begin{itemize}
 \item Erstellen eines Referenzcomputers
 \item Wechseln in eine Befehlszeile auf dem Referenzcomputer zum /Windows/System32/Sysprep oder den Ordner mit Sysprep.exe und Setupcl.exe.
 \item Beim Neustart des Computers durch Drücken der Taste F12 einen Netzwerkstart ausführen.
 \item Im Startmenü das zuvor erstellte Boot-Aufzeichnungsabbild ausw\"ahlen.
 \item Klicken auf Durchsuchen neben dem Namen und Speicherort und navigieren zu einem lokalen Ordner zur Speicherung.
 \item Angabe eines Namens für das Bild in der WIM-Erweiterung und speichern.
 \item Image hochladen ausw\"ahlen und verbinden.
 \item Wenn verlangt Benutzernamen und Kennwort für ein Konto mit Berechtigungen zum Verbinden angeben.
 \item Namen der Abbildgruppe angeben und fertigstellen.
\end{itemize}
\end{document}