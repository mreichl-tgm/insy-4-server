\documentclass[a4paper,11pt]{article}
\usepackage[utf8]{inputenc}
\usepackage[ngerman]{babel}
\usepackage{microtype}
\usepackage{cmbright}
\usepackage[T1]{fontenc}

%opening
\title{Windows Bereitstellungsdienste}
\author{Markus Reichl}

\begin{document}

\maketitle

\section{Einf\"uhrung}
Windows-Bereitstellungsdienste können Windows-Betriebssysteme bereitstellen. 
Diese Dienste können verwendet werden, um neue Computer mithilfe einer netzwerkbasierten Installation einzurichten.

\section{Installation}
\subsection{Vorbereitung}
Die Windows Bereitstellungsdienste ben\"otigen zumindest die Funktionen ``NTFS-Volume'', ``DNS'', ``DHCP'', ``Anmeldeinformationen''.

\subsection{Vorgehen}
\begin{itemize}
 \item Klick im Server-Manager auf verwalten.
 \item Klick auf Rollen und Features hinzufügen.
 \item Rollenbasierte oder featurebasierte Installation undServer zum Bereitstellen von WDS w\"ahlen.
 \item Auf der Serverrollen Seite Windows-Bereitstellungsdienste ausw\"ahlen.
 \item Klicken auf weiter.
\end{itemize}

Zus\"atzlich k\"onnen zwei Rollendienste installiert werden. ``Bereitstellungsserver'' und ``Transportserver'' installieren.

\subsubsection{Transport Server}
Diese Option bietet eine Teilmenge der Funktionen der Windows-Bereitstellungsdienste. Sie enthält nur die zentralen Netzwerkkomponenten.

Transport-Server können multicast-Namespaces zu erstellen, die Übertragung von Daten (einschließlich Betriebssystemabbildern) von einem eigenständigen Server.

\subsubsection{Deployment Server}
Diese Option bietet die volle Funktionalität von Windows Deployment Services, die zum Konfigurieren und Remoteinstallation von Windows-Betriebssystemen verwendet werden können. 

\section{Konfiguration}
Nach der Installation der Windows-Bereitstellungsdienst-Rolle muss der Server konfiguriert werden. 

Wenn Windows-Bereitstellungsdienste in Active Directory integriert ist, muss der Server Mitglied einer Active Directory-Domänendienste (AD DS)-Domäne oder ein Domänencontroller für eine AD DS-Domäne sein. 

Wenn Windows-Bereitstellungsdienste im eigenständigen Modus installiert ist, können sie ohne eine Abhängigkeit von Active Directory konfiguriert werden.

\subsection{Vorraussetzungen}
Es ist ein aktiver DHCP-Server auf die in Windows-Bereitstellungsdienste verwendet Pre-Execution Environment (PXE), die DHCP für die IP-Adressierung benötigt.

Es ist ein aktiver DNS-Server im Netzwerk. Der Server hat eine NTFS-Dateisystempartition auf dem Bilder gespeichert werden.

\subsection{Vorgehen}
Es gibt zwei Konfigurationsoptionen für die Windows-Bereitstellungsdienste. 
Es kann konfiguriert werden, um Active Directory integriert oder als eigenständiger Server konfiguriert werden.

Der Beispielserver verwendet den eigenst\"andigen Modus.
\begin{itemize}
 \item Klicken auf Tools und auf Windows-Bereitstellungsdienste.
 \item Auswahl des gew\"unschten Servers.
 \item Auswahl eigenst\"andiger Server.
 \item Standardpfad ausw\"ahlen, oder eigenen Pfad hinzufügen.
\end{itemize}

Wenn die Konfiguration löschen abgeschlossen ist k\"onnen nun die Images hinzufügt werden.
Wenn die Einstellungen des Servers geändert werden sollen, kann \"uber die Option ``Eigenschaften'' zugegriffen werden.
Nun muss zumindest ein Startabbild (die startbare Umgebung, mit dem ursprünglich gestartet, der Computer und ein Installationsabbild (die eigentlichen Abbilder, die Sie bereitstellen werden) angegeben werden.
Diese Funktion findet sich unter ``hinzufügen von Bildern''.

\end{document}
